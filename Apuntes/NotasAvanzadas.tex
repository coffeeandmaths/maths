\documentclass[12pt]{article}
% --------- Minimal page and spacing setup ---------
\usepackage[margin=2cm]{geometry}
\usepackage{setspace}
\setstretch{1.05} % tighter than default
\usepackage{parskip}
\setlength{\parskip}{0.4em}
\setlength{\parindent}{0pt}
\usepackage{import} % required to include external TeX

% --------- Math and Fonts ----------
\usepackage{amsmath,amssymb,amsfonts}
\usepackage{mathrsfs}  % for script letters
\usepackage{eucal}     % elegant calligraphic fonts
\usepackage{microtype} % better spacing
\usepackage{tikz}
\usetikzlibrary{cd} % for commutative diagrams

% --------- Section Style ----------
\usepackage{titlesec}
\titleformat{\section}{\large\bfseries}{\thesection.}{0.5em}{}
\titleformat{\subsection}{\normalsize\bfseries}{\thesubsection}{0.5em}{}

% --------- Header (optional) ----------
\usepackage{fancyhdr}
\pagestyle{fancy}
\fancyhead[L]{\small Notes}
\fancyhead[R]{\small \today}
\fancyfoot[C]{\thepage}

%---------- Centered Section Titles ----------
% Center section titles
\usepackage{titlesec}
\titleformat{\section}[block]{\centering\bfseries\Large}{\thesection}{1em}{}
\titleformat{\subsection}[block]{\centering\bfseries\large}{\thesubsection}{1em}{}
\titleformat{\subsubsection}[block]{\centering\bfseries\normalsize}{\thesubsubsection}{1em}{}

%---------- Basic Symbols ----------
% Blackboard bold number sets
\newcommand{\N}{\mathbb{N}}  % Natural numbers
\newcommand{\Z}{\mathbb{Z}}  % Integers
\newcommand{\Q}{\mathbb{Q}}  % Rationals
\newcommand{\R}{\mathbb{R}}  % Reals
\newcommand{\C}{\mathbb{C}}  % Complex numbers
\newcommand{\F}{\mathbb{F}}  % Finite field or generic field

% --------- Document Starts ----------
\begin{document}

When a function $f:G \Rightarrow H$ between two \textit{groups} preserves the group operation of each of the two groups it's called \textbf{homoforphism}.
It is irrelevant whether the operations in the each  groups are the same or not, the homomorphism \textit{transforms} them.
\\
\textbf{Definition} : Let the \textit{group homomorphism} $f:G \Rightarrow H$ and $a \in G$, $b \in H$ then
$$f(ab)=f(a)+f(b) \quad \forall a,b \in G$$
The full idea is that 
$$f:(G,*) \rightarrow (H,\cdot)$$
presetves structure but simply written as $f:G \Rightarrow H$ as the homomorphism uses $G$ and $H$
\begin{itemize}
	\item In group $G$
	$$m_G : G \times G, \quad m_g(a,b) := a*b$$
	\item In group $H$
	$$m_H : H \times H, \quad m_h(x,y) := x \cdot y$$
\end{itemize}
\vspace{1cm}
$$
\begin{tikzcd}
	G \times G \arrow[r, "f \times f"] \arrow[d, "m_G"'] & H \times H \arrow[d, "m_H"] \\
	G \arrow[r, "f"] & H
\end{tikzcd}

$$
\vspace{1cm} \\
$f \times f : G \times G \rightarrow H \times H$ is function defined as $(f \times f)(a,b) := (f(a),f(b))$ which takes a \textit{pair} of elements from $G \times G$ and then applies the function $f$ \textbf{to each component} and returns the corresponding pair in $H \times H$. Both paths starting at $G \times G$ arrive at $H$, that is preserving the structure.




\end{document}
