\documentclass[12pt]{article}
% --------- Minimal page and spacing setup ---------
\usepackage[margin=2cm]{geometry}
\usepackage{setspace}
\setstretch{1.05} % tighter than default
\usepackage{parskip}
\setlength{\parskip}{0.4em}
\setlength{\parindent}{0pt}
\usepackage{import} % required to include external TeX

% --------- Math and Fonts ----------
\usepackage{amsmath,amssymb,amsfonts}
\usepackage{mathrsfs}  % for script letters
\usepackage{eucal}     % elegant calligraphic fonts
\usepackage{microtype} % better spacing
\usepackage{tikz}
\usetikzlibrary{cd} % for commutative diagrams

% --------- Section Style ----------
\usepackage{titlesec}
\titleformat{\section}{\large\bfseries}{\thesection.}{0.5em}{}
\titleformat{\subsection}{\normalsize\bfseries}{\thesubsection}{0.5em}{}

% --------- Header (optional) ----------
\usepackage{fancyhdr}
\pagestyle{fancy}
\fancyhead[L]{\small Notes}
\fancyhead[R]{\small \today}
\fancyfoot[C]{\thepage}

%---------- Centered Section Titles ----------
% Center section titles
\usepackage{titlesec}
\titleformat{\section}[block]{\centering\bfseries\Large}{\thesection}{1em}{}
\titleformat{\subsection}[block]{\centering\bfseries\large}{\thesubsection}{1em}{}
\titleformat{\subsubsection}[block]{\centering\bfseries\normalsize}{\thesubsubsection}{1em}{}

%---------- Basic Symbols ----------
% Blackboard bold number sets
\newcommand{\N}{\mathbb{N}}  % Natural numbers
\newcommand{\Z}{\mathbb{Z}}  % Integers
\newcommand{\Q}{\mathbb{Q}}  % Rationals
\newcommand{\R}{\mathbb{R}}  % Reals
\newcommand{\C}{\mathbb{C}}  % Complex numbers
\newcommand{\F}{\mathbb{F}}  % Finite field or generic field

% --------- Document Starts ----------
\begin{document}

\section*{Injective, Surjective, Bijective in the Context of Homomorphisms}

\subsection*{Domain, Codomain, and Image}

\begin{itemize}
	\item \textbf{Domain}: The group from which the homomorphism takes its input elements.
	\item \textbf{Codomain}: The group to which the homomorphism sends its outputs.
	\item \textbf{Image}: The subgroup of the codomain consisting of all elements of the form \( f(g) \) with \( g \in G \).
\end{itemize}

\bigskip

\begin{center}
	\begin{tabular}{| >{\raggedright\arraybackslash}m{2.8cm} 
			| >{\raggedright\arraybackslash}m{4.8cm} 
			| >{\raggedright\arraybackslash}m{6.5cm} 
			| >{\raggedright\arraybackslash}m{4.5cm} |}
		\hline
		\textbf{Property} & \textbf{Definition} & \textbf{Concrete Meaning} & \textbf{Example (Group Homomorphism)} \\
		\hline
		
		\textbf{Injective} & The kernel contains only the identity: \( \ker f = \{e_G\} \). 
		& No two distinct elements of the domain map to the same element of the codomain. 
		& \( f : \mathbb{Z} \to \mathbb{Z},\ f(n) = 2n \) under addition. \\
		
		\hline
		
		\textbf{Surjective} & \( \operatorname{im} f = H \) (the image equals the whole codomain). 
		& Every element of the codomain is hit by some element of the domain. 
		& \( f : \mathbb{Z} \to \mathbb{Z}_n,\ f(k) = k \ (\mathrm{mod}\ n) \) under addition. \\
		
		\hline
		
		\textbf{Bijective} & Both injective and surjective. 
		& A one-to-one correspondence between domain and codomain; \( f \) is an isomorphism of groups. 
		& \( f : \mathbb{Z}_n \to \langle g \rangle \subset G,\ f(k) = g^k \) where \( g \) has order \( n \). \\
		
		\hline
	\end{tabular}
\end{center}

\section*{Homomorphisms and Related Morphisms}

\begin{center}
	\renewcommand{\arraystretch}{1.4}
	\begin{tabular}{|c|c|p{7cm}|}
		\hline
		\textbf{Concept} & \textbf{In Groups} & \textbf{Key Property} \\
		\hline
		\textbf{Morphism} & Homomorphism \( f : G \to H \) & Preserves the group operation: \( f(a \cdot b) = f(a) \cdot f(b) \). \\
		\hline
		\textbf{Monomorphism} & Injective homomorphism & Left-cancellable: \( f \circ g_1 = f \circ g_2 \Rightarrow g_1 = g_2 \). Equivalent to \( \ker f = \{e_G\} \). \\
		\hline
		\textbf{Epimorphism} & Surjective homomorphism & Right-cancellable: \( h_1 \circ f = h_2 \circ f \Rightarrow h_1 = h_2 \). Image equals codomain. \\
		\hline
		\textbf{Isomorphism} & Bijective homomorphism & There exists \( g : H \to G \) such that \( g \circ f = \mathrm{id}_G \) and \( f \circ g = \mathrm{id}_H \). \\
		\hline
		\textbf{Automorphism} & Isomorphism \( f : G \to G \) & A bijective homomorphism from a group to itself (a symmetry of the group). \\
		\hline
	\end{tabular}
\end{center}

\subsubsection*{Symmetry of Structure}
An \textbf{automorphism} of a group is a bijective homomorphism from the group to itself, representing a structural symmetry. Examples include inner automorphisms in group theory, which conjugate every element by a fixed group element.


\section*{Homomorphism properties}
When a function $f:G \Rightarrow H$ between two \textit{groups} preserves the group operation of each of the two groups it's called \textbf{homomorphism}.
It is irrelevant whether the operations in the each  groups are the same or not, the homomorphism \textit{transforms} them.
\vspace{1cm}\\
\textbf{Definition} : Let the \textit{group homomorphism} $f:G \Rightarrow H$ and $a \in G$, $b \in H$ then
$$f(ab)=f(a)+f(b) \quad \forall a,b \in G$$
The full idea is that 
$$f:(G,*) \rightarrow (H,\cdot)$$
presetves structure but simply written as $f:G \Rightarrow H$ as the homomorphism uses $G$ and $H$
\begin{itemize}
	\item In group $G$
	$$m_G : G \times G, \quad m_g(a,b) := a*b$$
	\item In group $H$
	$$m_H : H \times H, \quad m_h(x,y) := x \cdot y$$
\end{itemize}
\vspace{1cm}
$$
\begin{tikzcd}
	G \times G \arrow[r, "f \times f"] \arrow[d, "m_G"'] & H \times H \arrow[d, "m_H"] \\
	G \arrow[r, "f"] & H
\end{tikzcd}

$$
\vspace{1cm} \\
$f \times f : G \times G \rightarrow H \times H$ is function defined as $(f \times f)(a,b) := (f(a),f(b))$ which takes a \textit{pair} of elements from $G \times G$ and then applies the function $f$ \textbf{to each component} and returns the corresponding pair in $H \times H$. Both paths starting at $G \times G$ arrive at $H$, that is preserving the structure.
\vspace{1cm} \\
\textbf{Definition} : a \textit{group homomorphism} preserves the identity element 
$$
f(e_G)=e_H
$$
\textit{proof :}

Since for any $a \in G$:

$$
e_G * a = a
$$

Apply $f$ to both sides:

$$
f(e_G * a) = f(a)
$$

Using the homomorphism property:

$$
f(e_G) \cdot f(a) = f(a)
$$

Cancel $f(a)$ (possible since $H$  is a group), and conclude:

$$
f(e_G) = e_H
$$
its commutative diagram :

$$
\begin{tikzcd}
	G \arrow[rr, "f"] & & H \\
	& G \arrow[ul, "e_G"'] \arrow[ur, "e_H"']
\end{tikzcd}
$$

\textbf{Definition}: a $group$ homomorphism also preserves inverses

$$
f(a^{-1}) = f(a)^{-1}
$$

\textit{proof}

From $group$ identity: 

$$
a * a^{-1} = e_G
$$

Apply $f$:

$$
f(a * a^{-1}) = f(e_G)
$$

By the homomorphism property:

$$
f(a) \cdot f(a^{-1}) = e_H
$$

So $f(a^{-1})$ is the inverse of $f(a)$, which gives:

$$
f(a^{-1}) = f(a)^{-1}
$$
the commutative diagram then is 
$$
\begin{tikzcd}
	G \arrow[r, "f"] \arrow[d, "\mathrm{inv}_G"'] & H \arrow[d, "\mathrm{inv}_H"] \\
	G \arrow[r, "f"] & H
\end{tikzcd}
$$

\section*{Canonical decompostition}
\subsection*{Subgroup}
Let $(G,*)$ be a group and $H \subseteq G$ a \textbf{subset} if only if :
\begin{itemize}
	\item Closure: $\forall a,b \in H, \quad a*b \in H$
	\item Identity : $e_g \in H$
	\item Inverse : $\forall a \in H, \quad a^{-1} \in H$
\end{itemize}
The \textbf{associativity} of the operation $*$ is also inherited from $G$
\subsection*{Cosets}
Because $H$ is inside $G$ to translate the whole subset through the group $G$ by multiplying a fixed element $g \in G$. Because in gneral , group multiplication \textit{may not be commutative} 
$$
gh \neq hg
$$
They are only equial in abelian groups.


$\textbf{Example: Left vs Right Cosets in a Non-Abelian Group}$

Let  
$$
G = S_3 = \{ e, (12), (13), (23), (123), (132) \}
$$
be the symmetric group on 3 elements, with the operation being permutation composition.  
Consider the subgroup:  
$$
H = \{ e, (12) \}.
$$

Let $g = (13)$.

$$
\text{Left coset: } gH = (13)H = \{ (13)e, \ (13)(12) \}.
$$
We compute:
$$
(13)e = (13), \quad (13)(12) = (132),
$$
so:
$$
gH = \{ (13), (132) \}.
$$


$$
\text{Right coset: } Hg = H(13) = \{ e(13), \ (12)(13) \}.
$$
We compute:
$$
e(13) = (13), \quad (12)(13) = (123),
$$
so:
$$
Hg = \{ (13), (123) \}.
$$

$\textbf{Comparison:}$  
$$
gH = \{ (13), (132) \} \quad \neq \quad Hg = \{ (13), (123) \}.
$$

Since $gH \neq Hg$, we see that in the non-abelian group $S_3$, left and right cosets of the same subgroup can differ.  
In an abelian group, we would always have $gH = Hg$.

\subsection*{Kernel}

Let $f: G \to H$ be a group homomorphism between groups $(G, *)$ and $(H, \cdot)$.  
The **kernel** of $f$ is defined as:
$$
\ker(f) = \{ g \in G \mid f(g) = e_H \}
$$
where $e_H$ is the identity element of $H$.

\textbf{Theorem}

The kernel $\ker(f)$ is a subgroup of $G$.


\textit{proof}

\textbf{Identity in the kernel} 
Let $e_G$ and $e_H$ be the identities of $G$ and $H$.  
Since $f$ is a homomorphism:
$$
f(e_G) = f(e_G * e_G) = f(e_G) \cdot f(e_G)
$$
Cancelling $f(e_G)$ : $f(e_G) = e_H$

Start from the homomorphism property and the identity law in \(G\):
$$
e_G * e_G = e_G \quad \Rightarrow \quad f(e_G * e_G) = f(e_G).
$$
Since \(f\) is a homomorphism,
$$
f(e_G * e_G) = f(e_G)\cdot f(e_G).
$$
Thus we have the equation in \(H\):
$$
f(e_G) = f(e_G)\cdot f(e_G).
$$

We know that in $G$:
$$
e_G * e_G = e_G.
$$
Applying $f$ to both sides:
$$
f(e_G * e_G) = f(e_G).
$$
Since $f$ is a homomorphism:
$$
f(e_G * e_G) = f(e_G) \cdot f(e_G).
$$
Thus in $H$ we have:
$$
f(e_G) = f(e_G) \cdot f(e_G).
$$
Multiply on the left by $f(e_G)^{-1}$ (which exists because $H$ is a group):
$$
f(e_G)^{-1} \cdot f(e_G) = f(e_G)^{-1} \cdot \big(f(e_G) \cdot f(e_G)\big).
$$
By associativity in $H$:
$$
e_H = \big(f(e_G)^{-1} \cdot f(e_G)\big) \cdot f(e_G) = e_H \cdot f(e_G) = f(e_G).
$$
Therefore:
$$
f(e_G) = e_H.
$$

$$
f(e_G) = e_H.
$$

(Equivalently, you could multiply \textit{on the right} by $f(e_g)^{-1}$ and reach the same conclusion.), thus 
$$
e_G \in \ker(f)
. The$$


\textbf{Closure under the group operation}  

Let $a, b \in \ker(f)$. Then:
$$
f(a) = e_H, \quad f(b) = e_H
$$
Using the homomorphism property:
$$
f(a * b) = f(a) \cdot f(b) = e_H \cdot e_H = e_H
$$
Thus $a * b \in \ker(f)$.

\textbf{Closure under inverses}
  
Let $a \in \ker(f)$. Then $f(a) = e_H$.  
Using the property for inverses:
$$
f(a^{-1}) = f(a)^{-1} = e_H^{-1} = e_H
$$
Thus $a^{-1} \in \ker(f)$.

Since $\ker(f)$ contains the identity, is closed under the group operation, and is closed under inverses, we conclude:
$$
\ker(f) \leq G
$$
$\square$

\textit{One-Line Subgroup Test}

A quicker proof uses the subgroup test:  
If $a, b \in \ker(f)$ then:
$$
f(a b^{-1}) = f(a) \cdot f(b^{-1}) = f(a) \cdot f(b)^{-1} = e_H \cdot e_H^{-1} = e_H
$$
Thus $a b^{-1} \in \ker(f)$, so $\ker(f) \leq G$.

\subsection*{Partition $G /ker(f)$}


Let  $f : G \to H$ be a group homomorphism and define the relation on $G$ by: $g_1 \sim g_2 \; \Longleftrightarrow \; g_1^{-1} g_2 \in \ker(f)$
$$
g_1 \sim g_2 \iff f(g_1) = f(g_2).
$$

The equivalence class of $g \in G$ is $$[g] = g \ker(f) = \{ gk \mid k \in \ker(f) \}$$
and the set of all the $[g]$ partitions is
$$
G / \ker(f) \;=\; \{ [g] \mid g \in G \} \;=\; G / \!\sim,
$$

and since  $\ker(f) \trianglelefteq G, \; G / \ker(f)$  is a group with:

$$
(g_1 \ker(f)) \cdot (g_2 \ker(f)) \;=\; (g_1 g_2) \ker(f).
$$


\textit{Example} $\quad f : (\mathbb{Z}, +) \ \longrightarrow \ (\mathbb{Z}_6, +_6)$, 
$f(k) = 2k \ (\mathrm{mod}\ 6)$

where:
\begin{itemize}
	\item $(\mathbb{Z}, +)$ is the group of integers under usual addition.
	\item $(\mathbb{Z}_6, +_6)$ is the group of integers modulo $6$ under addition mod $6$.
\end{itemize}


\textbf{Homomorphism property:} 
$$
\forall k, \ell \in \mathbb{Z}, \ 
f(k + \ell) = 2(k + \ell) \ (\mathrm{mod}\ 6) 
= (2k + 2\ell) \ (\mathrm{mod}\ 6) 
= f(k) +_6 f(\ell)
$$

\textbf{Kernel:}

$$\ker f = \{ k \in \mathbb{Z} \ : \ 2k \equiv 0 \ (\mathrm{mod}\ 6) \}
= \{ k \in \mathbb{Z} \ : \ 3 \mid k \}
= 3\mathbb{Z}
$$


\textbf{Partition of } $G = \mathbb{Z} \ \text{by } \ker f:$

$$
0 + 3\mathbb{Z}, \quad 1 + 3\mathbb{Z}, \quad 2 + 3\mathbb{Z}
$$
$$
\begin{aligned}
	0 + 3\mathbb{Z} &= \{\dots,-6,-3,0,3,6,\dots\}, \\
	1 + 3\mathbb{Z} &= \{\dots,-5,-2,1,4,7,\dots\}, \\
	2 + 3\mathbb{Z} &= \{\dots,-4,-1,2,5,8,\dots\}.
\end{aligned}
$$

These cosets are pairwise disjoint and their union is $\mathbb{Z}$.


\textbf{Quotient group:} 
$$
\mathbb{Z} / \ker f = \mathbb{Z} / 3\mathbb{Z} 
= \{ \ 0 + 3\mathbb{Z}, \ 1 + 3\mathbb{Z}, \ 2 + 3\mathbb{Z} \ \}
$$


\textbf{Image:}
$$
\operatorname{im} f = \{0, 2, 4\} \subseteq \mathbb{Z}_6
$$
which is the subgroup generated by $2$ in $\mathbb{Z}_6$, isomorphic to $\mathbb{Z}_3$.

\textbf{Induced isomorphism:}
$$
\pi : \mathbb{Z} \to \mathbb{Z} / 3\mathbb{Z}, \ 
\pi(k) = k + 3\mathbb{Z}
$$
Define
$$
\tilde{f} : \mathbb{Z} / 3\mathbb{Z} \ \longrightarrow \ \operatorname{im} f, 
\quad \tilde{f}(r + 3\mathbb{Z}) = 2r \ (\mathrm{mod}\ 6)
$$
This is a well-defined bijective homomorphism, so:
$$
\mathbb{Z} / \ker f \ \cong \ \operatorname{im} f
$$
the commutative diagrame is

$$
\begin{tikzcd}[row sep=large, column sep=large]
	(\mathbb{Z}, +) \arrow[r, "f"] \arrow[d, "\pi"'] 
	& (\mathbb{Z}_6, +_6) \\
	\mathbb{Z}/3\mathbb{Z} \arrow[r, "\tilde{f}"] 
	& \operatorname{im}(f) \arrow[u, hook, "\iota"]
\end{tikzcd}
$$

Example: Let $ \operatorname{sgn} : S_3 \longrightarrow \{\pm 1\}$ be the sign homomorphism from the symmetric group on three elements to the multiplicative group $\{\pm 1\}$.
\vspace{0.5cm} \\
\textbf{Homomorphism property}: For all $\sigma, \tau \in S_3$,

$$
\operatorname{sgn}(\sigma \tau) = \operatorname{sgn}(\sigma) \cdot \operatorname{sgn}(\tau).
$$

\textbf{Kernel:}
$$
\ker(\operatorname{sgn}) = A_3 = \{ e,\ (123),\ (132) \}.
$$

\textbf{Partition }of $S_3$ by $\ker(\operatorname{sgn})$:
$$
A_3 = \{ e, (123), (132) \}, \quad (12)A_3 = \{ (12), (13), (23) \}.
$$
These two cosets are disjoint and their union is $S_3$.

\textbf{Quotient group:}
$$
S_3 / A_3 = \{\, A_3,\ (12)A_3 \,\} \cong \mathbb{Z}_2.
$$

\textbf{Image}
$$
\operatorname{im}(\operatorname{sgn}) = \{\pm 1\} \cong \mathbb{Z}_2.
$$

\textbf{Induced isomorphism:} Let $\pi : S_3 \to S_3/A_3$ be the natural projection

$$
\pi(\sigma) = \sigma A_3.
$$
Define

$$
\widetilde{\operatorname{sgn}} : S_3 / A_3 \longrightarrow \{\pm 1\}, \quad 
\widetilde{\operatorname{sgn}}(A_3) = +1, \quad \widetilde{\operatorname{sgn}}((12)A_3) = -1.
$$
This is a well-defined bijective homomorphism, so
$$
S_3 / \ker(\operatorname{sgn}) \ \cong \ \operatorname{im}(\operatorname{sgn}).
$$

$$
\begin{tikzcd}[row sep=large, column sep=large]
	S_3 \arrow[r, "\operatorname{sgn}"] \arrow[d, "\pi"'] 
	& \{\pm 1\} \\
	S_3 / A_3 \arrow[r, "\widetilde{\operatorname{sgn}}"] 
	& \operatorname{im}(\operatorname{sgn}) \arrow[u, hook, "\iota"]
\end{tikzcd}
$$


The full decomposition of a \textit{morphism} 

$$
G \xrightarrow{\pi} G/\ker(f) \xrightarrow{f^{\sim}} \operatorname{im}(f) \hookrightarrow H
$$













\end{document}
