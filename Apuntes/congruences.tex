% Options for packages loaded elsewhere
\PassOptionsToPackage{unicode}{hyperref}
\PassOptionsToPackage{hyphens}{url}
\documentclass[
]{article}
\usepackage{xcolor}
\usepackage{amsmath,amssymb}
\setcounter{secnumdepth}{-\maxdimen} % remove section numbering
\usepackage{iftex}
\ifPDFTeX
  \usepackage[T1]{fontenc}
  \usepackage[utf8]{inputenc}
  \usepackage{textcomp} % provide euro and other symbols
\else % if luatex or xetex
  \usepackage{unicode-math} % this also loads fontspec
  \defaultfontfeatures{Scale=MatchLowercase}
  \defaultfontfeatures[\rmfamily]{Ligatures=TeX,Scale=1}
\fi
\usepackage{lmodern}
\ifPDFTeX\else
  % xetex/luatex font selection
\fi
% Use upquote if available, for straight quotes in verbatim environments
\IfFileExists{upquote.sty}{\usepackage{upquote}}{}
\IfFileExists{microtype.sty}{% use microtype if available
  \usepackage[]{microtype}
  \UseMicrotypeSet[protrusion]{basicmath} % disable protrusion for tt fonts
}{}
\makeatletter
\@ifundefined{KOMAClassName}{% if non-KOMA class
  \IfFileExists{parskip.sty}{%
    \usepackage{parskip}
  }{% else
    \setlength{\parindent}{0pt}
    \setlength{\parskip}{6pt plus 2pt minus 1pt}}
}{% if KOMA class
  \KOMAoptions{parskip=half}}
\makeatother
\usepackage{longtable,booktabs,array}
\usepackage{calc} % for calculating minipage widths
% Correct order of tables after \paragraph or \subparagraph
\usepackage{etoolbox}
\makeatletter
\patchcmd\longtable{\par}{\if@noskipsec\mbox{}\fi\par}{}{}
\makeatother
% Allow footnotes in longtable head/foot
\IfFileExists{footnotehyper.sty}{\usepackage{footnotehyper}}{\usepackage{footnote}}
\makesavenoteenv{longtable}
\setlength{\emergencystretch}{3em} % prevent overfull lines
\providecommand{\tightlist}{%
  \setlength{\itemsep}{0pt}\setlength{\parskip}{0pt}}
\usepackage{bookmark}
\IfFileExists{xurl.sty}{\usepackage{xurl}}{} % add URL line breaks if available
\urlstyle{same}
\hypersetup{
  hidelinks,
  pdfcreator={LaTeX via pandoc}}

\author{}
\date{}

\begin{document}

\subsection{Basic Definition}\label{basic-definition}

Let \(a, b \in \mathbb{Z}\) and \(n \in \mathbb{N}\), \(n \geq 2\).\\
We say:

\[
a \equiv b \pmod{n} \quad \Leftrightarrow \quad n \mid (a - b)
\]

That is, \(a\) and \(b\) leave the same remainder when divided by \(n\).

\subsection{Constructing Classes from
Division}\label{constructing-classes-from-division}

For any \(a \in \mathbb{Z}\), by the Division Algorithm, there exist
unique \(q, r \in \mathbb{Z}\) such that:

\[
a = nq + r, \quad 0 \leq r < n
\]

Then:

\[
a \equiv r \pmod{n} \quad \Rightarrow \quad \overline{a} = \overline{r}, \quad [a] = [r]
\]

So any integer \(a\) belongs to the same congruence class as its
remainder mod \(n\).

\subsection{Congruence Classes = Equivalence Classes
}\label{congruence-classes-equivalence-classes}

The relation \(\equiv \pmod{n}\) is an \textbf{equivalence relation} on
\(\mathbb{Z}\):

\begin{itemize}
\tightlist
\item
  \textbf{Reflexive}: \(a \equiv a\)
\item
  \textbf{Symmetric}: \(a \equiv b \Rightarrow b \equiv a\)
\item
  \textbf{Transitive}: \(a \equiv b\) and
  \(b \equiv c \Rightarrow a \equiv c\)
\end{itemize}

Therefore, each congruence class (written \(\overline{a}\) or \([a]\))
is an \textbf{equivalence class}.

\subsection{Notations}\label{notations}

\begin{longtable}[]{@{}
  >{\raggedright\arraybackslash}p{(\linewidth - 4\tabcolsep) * \real{0.1569}}
  >{\raggedright\arraybackslash}p{(\linewidth - 4\tabcolsep) * \real{0.3922}}
  >{\raggedright\arraybackslash}p{(\linewidth - 4\tabcolsep) * \real{0.4510}}@{}}
\toprule\noalign{}
\begin{minipage}[b]{\linewidth}\raggedright
Notation
\end{minipage} & \begin{minipage}[b]{\linewidth}\raggedright
Author / Context
\end{minipage} & \begin{minipage}[b]{\linewidth}\raggedright
Example
\end{minipage} \\
\midrule\noalign{}
\endhead
\bottomrule\noalign{}
\endlastfoot
\(\overline{a}\) & Dummit \& Foote (group and ring theory) &
\(\overline{7} = \overline{2} \in \mathbb{Z}_5\) \\
\([a]\) or \([a]_n\) & Aluffi (category and structural algebra) &
\([7] = [2] \in \mathbb{Z}/5\mathbb{Z}\) \\
\end{longtable}

\subsection{\texorpdfstring{Construction of
\(\mathbb{Z}/n\mathbb{Z}\)}{Construction of \textbackslash mathbb\{Z\}/n\textbackslash mathbb\{Z\}}}\label{construction-of-mathbbznmathbbz}

\subsubsection{Dummit's Approach
(Group-Theoretic):}\label{dummits-approach-group-theoretic}

\begin{enumerate}
\def\labelenumi{\arabic{enumi}.}
\item
  Define relation on \(\mathbb{Z}\):\\
  \(a \sim b\) if \(a \equiv b \pmod{n}\)
\item
  This partitions \(\mathbb{Z}\) into \(n\) equivalence classes:\\
  \(\overline{0}, \overline{1}, \dots, \overline{n-1}\)
\item
  Define:

  \[
  \mathbb{Z}_n = \mathbb{Z}/n\mathbb{Z} = \{ \overline{0}, \overline{1}, \dots, \overline{n-1} \}
  \]
\item
  Define operations:

  \begin{itemize}
  \tightlist
  \item
    \(\overline{a} + \overline{b} := \overline{a + b}\)
  \item
    \(\overline{a} \cdot \overline{b} := \overline{ab}\)
  \end{itemize}
\end{enumerate}

\subsubsection{Aluffi's Approach (Categorical / Quotient
Object):}\label{aluffis-approach-categorical-quotient-object}

\begin{enumerate}
\def\labelenumi{\arabic{enumi}.}
\item
  Start from the ring \(\mathbb{Z}\) and the ideal \((n) = n\mathbb{Z}\)
\item
  Define:

  \[
  [a]_n = a + n\mathbb{Z} = \{ a + nk \mid k \in \mathbb{Z} \}
  \]
\item
  The quotient ring is:

  \[
  \mathbb{Z}/n\mathbb{Z} = \{ [0], [1], \dots, [n-1] \}
  \]
\item
  Operations are defined via coset representatives:

  \begin{itemize}
  \tightlist
  \item
    \([a] + [b] := [a + b]\)
  \item
    \([a] \cdot [b] := [ab]\)
  \end{itemize}
\end{enumerate}

Both approaches define \textbf{the same ring} --- the ring of integers
modulo \(n\).

\subsection{Modular Operations}\label{modular-operations}

For \(\overline{a}, \overline{b} \in \mathbb{Z}_n\) or
\([a], [b] \in \mathbb{Z}/n\mathbb{Z}\):

\begin{itemize}
\tightlist
\item
  \textbf{Addition}:
  $\overline{a} + \overline{b} = \overline{a + b}\), \([a] + [b] = [a + b]$
\item
  \textbf{Multiplication}:
  \(\overline{a} \cdot \overline{b} = \overline{ab}\), \([a] \cdot [b] = [ab]\)
\item
  \textbf{Exponentiation}:
  \(\overline{a}^k = \overline{a^k}\), \([a]^k = [a^k]\)
\item
  \textbf{Inverses}: \(\overline{a}^{-1}\) exists iff \(\gcd(a, n) = 1\)
\end{itemize}

\subsection{\texorpdfstring{Example:
\(\mathbb{Z}_5\)}{Example: \textbackslash mathbb\{Z\}\_5}}\label{example-mathbbz_5}

\begin{itemize}
\tightlist
\item
  \(\overline{2} = \{ ..., -8, -3, 2, 7, 12, ... \} = [2]\)
\item
  \(\overline{3} + \overline{4} = \overline{7} = \overline{2}\)
\end{itemize}

\subsection{\texorpdfstring{Algebraic Structures in
\(\mathbb{Z}_n\)}{Algebraic Structures in \textbackslash mathbb\{Z\}\_n}}\label{algebraic-structures-in-mathbbz_n}

\begin{longtable}[]{@{}
  >{\raggedright\arraybackslash}p{(\linewidth - 4\tabcolsep) * \real{0.0857}}
  >{\raggedright\arraybackslash}p{(\linewidth - 4\tabcolsep) * \real{0.5619}}
  >{\raggedright\arraybackslash}p{(\linewidth - 4\tabcolsep) * \real{0.3524}}@{}}
\toprule\noalign{}
\begin{minipage}[b]{\linewidth}\raggedright
Structure
\end{minipage} & \begin{minipage}[b]{\linewidth}\raggedright
Description
\end{minipage} & \begin{minipage}[b]{\linewidth}\raggedright
Properties
\end{minipage} \\
\midrule\noalign{}
\endhead
\bottomrule\noalign{}
\endlastfoot
\textbf{Group} & \((\mathbb{Z}_n, +)\) & Finite abelian group \\
\textbf{Ring} & \((\mathbb{Z}_n, +, \cdot)\) & Commutative ring with
identity \\
\textbf{Field} & \(\mathbb{Z}_p\) if \(p\) is prime & All nonzero
\(\overline{a}\) invertible \\
\textbf{Units} &
\(\mathbb{Z}_n^\times = \{ \overline{a} \mid \gcd(a,n)=1 \}\) &
Multiplicative group \\
\end{longtable}

\subsection{Worked Examples}\label{worked-examples}

\subsubsection{\texorpdfstring{1. In
\(\mathbb{Z}_6\):}{1. In \textbackslash mathbb\{Z\}\_6:}}\label{in-mathbbz_6}

\begin{itemize}
\tightlist
\item
  \(\overline{4} + \overline{5} = \overline{9} = \overline{3}\)
\item
  \([2] \cdot [4] = [8] = [2]\)
\end{itemize}

\subsubsection{\texorpdfstring{2. Inverse in
\(\mathbb{Z}_7\):}{2. Inverse in \textbackslash mathbb\{Z\}\_7:}}\label{inverse-in-mathbbz_7}

\begin{itemize}
\tightlist
\item
  \(\overline{3} \cdot \overline{5} = \overline{1} \Rightarrow \overline{5}^{-1} = \overline{3}\)
\end{itemize}

\subsection{Modular Arithmetic as Set
Limiter}\label{modular-arithmetic-as-set-limiter}

Modular arithmetic \textbf{wraps} \(\mathbb{Z}\) into a finite cycle:

\begin{itemize}
\tightlist
\item
  \(\mathbb{Z} \to\) infinite
\item
  \(\mathbb{Z}_n = \{ \overline{0}, \dots, \overline{n-1} \}\)
\end{itemize}

Useful for: - Clocks → \(\mathbb{Z}_{12}\) - Days of the week →
\(\mathbb{Z}_7\) - Binary logic → \(\mathbb{Z}_2\) \#\# 🚀 Fast Modular
Exponentiation

To compute \(7^{222} \mod 13\):

Use \textbf{binary exponentiation}:

\begin{itemize}
\tightlist
\item
  \(7^2 \equiv 10\), \(7^4 \equiv 9\), \(7^8 \equiv 3\), etc.
\item
  Reduce mod 13 at each step to avoid overflow \#\# 🧰 Applications
\end{itemize}

\begin{longtable}[]{@{}
  >{\raggedright\arraybackslash}p{(\linewidth - 2\tabcolsep) * \real{0.2529}}
  >{\raggedright\arraybackslash}p{(\linewidth - 2\tabcolsep) * \real{0.7471}}@{}}
\toprule\noalign{}
\begin{minipage}[b]{\linewidth}\raggedright
Area
\end{minipage} & \begin{minipage}[b]{\linewidth}\raggedright
Use
\end{minipage} \\
\midrule\noalign{}
\endhead
\bottomrule\noalign{}
\endlastfoot
\textbf{Cryptography} & RSA, ECC → operations in \(\mathbb{Z}_n\) and
\(\mathbb{Z}_p^\times\) \\
\textbf{Hashing} & Use \(h(x) = x \mod m\) to map data \\
\textbf{Computer Systems} & Integers wrap mod \(2^n\) (overflow) \\
\textbf{Number Theory} & Fermat's little theorem, Chinese Remainder
Theorem \\
\textbf{Time/Calendars} & Model cyclic behavior with \(\mod n\) \\
\textbf{Digital Signatures} & Use exponentiation in mod fields \\
\end{longtable}

\subsection{💡 Final Insight}\label{final-insight}

\begin{itemize}
\tightlist
\item
  \(\overline{a}\) (Dummit) and \([a]\) (Aluffi) both describe
  \textbf{equivalence classes} in \(\mathbb{Z}/n\mathbb{Z}\)
\item
  Modular arithmetic is more than remainders: it's about working in
  \textbf{quotient structures}, a key idea across all of abstract
  algebra.
\end{itemize}

\section{🧮 Quotient Groups --- Aluffi's Definition and Core
Concepts}\label{quotient-groups-aluffis-definition-and-core-concepts}

\subsection{🔧 What Is a Quotient?}\label{what-is-a-quotient}

A \textbf{quotient} is a way to build a new structure by:

\begin{itemize}
\tightlist
\item
  \textbf{Identifying} elements that are declared equivalent (via an
  equivalence relation)
\item
  Then \textbf{collapsing} each group of equivalent elements into a
  single object in the new structure
\end{itemize}

\subsubsection{🔁 Clarifying the Terms:}\label{clarifying-the-terms}

\begin{longtable}[]{@{}
  >{\raggedright\arraybackslash}p{(\linewidth - 2\tabcolsep) * \real{0.1379}}
  >{\raggedright\arraybackslash}p{(\linewidth - 2\tabcolsep) * \real{0.8621}}@{}}
\toprule\noalign{}
\begin{minipage}[b]{\linewidth}\raggedright
Term
\end{minipage} & \begin{minipage}[b]{\linewidth}\raggedright
Meaning
\end{minipage} \\
\midrule\noalign{}
\endhead
\bottomrule\noalign{}
\endlastfoot
\textbf{Identify} & Declare two elements ``equivalent'' under a
relation \\
\textbf{Collapse} & Replace a whole class of equivalent elements with
one object (e.g.~a coset) \\
\end{longtable}

In quotient groups, we \textbf{identify} elements using a rule, then
\textbf{collapse} those into \textbf{cosets}, which form the elements of
the new group. \#\# 📘 Aluffi's Definition

Let \(G\) be a group and \(N \trianglelefteq G\) (a \textbf{normal
subgroup}).\\
Define a relation on \(G\) by:

\[
a \sim b \quad \text{if and only if} \quad a^{-1}b \in N
\]

This is an \textbf{equivalence relation} on \(G\), and each equivalence
class is the \textbf{left coset}:

\[
[a] = aN = \{ an \mid n \in N \}
\]

Then:

\[
G / N = \{ aN \mid a \in G \}
\]

with group operation:

\[
(aN)(bN) = (ab)N
\]

This is the \textbf{quotient group} \(G/N\).

\subsection{🔒 What Is a Normal
Subgroup?}\label{what-is-a-normal-subgroup}

A subgroup \(N \subseteq G\) is called \textbf{normal} if:

\[
gN = Ng \quad \text{for all } g \in G
\]

Equivalently:

\[
gng^{-1} \in N \quad \text{for all } g \in G,\ n \in N
\]

Why we need normality: - It ensures that the coset product
\((aN)(bN) = (ab)N\) is \textbf{well-defined} - Without normality, this
multiplication \textbf{depends on the representatives}

\subsection{\texorpdfstring{🧮 Example:
\(\mathbb{Z} / n\mathbb{Z}\)}{🧮 Example: \textbackslash mathbb\{Z\} / n\textbackslash mathbb\{Z\}}}\label{example-mathbbz-nmathbbz}

Let \(G = \mathbb{Z}\) and \(N = n\mathbb{Z}\).

\begin{itemize}
\tightlist
\item
  \(n\mathbb{Z}\) is a subgroup (and normal, since \(\mathbb{Z}\) is
  abelian)
\item
  Define \(a \sim b\) if \(a - b \in n\mathbb{Z}\) (i.e.,
  \(a \equiv b \mod n\))
\item
  The equivalence class of \(a\) is: \[
  [a] = a + n\mathbb{Z} = \{ a + kn \mid k \in \mathbb{Z} \}
  \]
\item
  The quotient group is: \[
  \mathbb{Z} / n\mathbb{Z} = \{ [0], [1], \dots, [n - 1] \}
  \]
\item
  With group operation: \([a] + [b] = [a + b]\)
\end{itemize}

This is the familiar \textbf{modular arithmetic group} \(\mathbb{Z}_n\).
\#\# ✅ Summary

\begin{itemize}
\tightlist
\item
  A \textbf{quotient group} \(G/N\) is formed by identifying elements
  via a normal subgroup \(N\) and collapsing them into \textbf{cosets}
\item
  \textbf{Aluffi's definition} uses equivalence: \(a \sim b\) if
  \(a^{-1}b \in N\)
\item
  \textbf{Cosets} become the elements of the new group
\item
  \textbf{Normality} is essential for the quotient to behave like a
  group
\item
  Example: \(\mathbb{Z}/n\mathbb{Z}\) collapses all integers congruent
  mod \(n\)
\end{itemize}

\subsection{Preliminaries}\label{preliminaries}

\textbf{Modular Congruence}:

Formal Mathematical Definition

We define:

\[a \equiv b \pmod{n} \iff n \mid (a - b)\]

This means: the difference between \(a\) and \(b\) is divisible by
\(n\), there exists some integer \(k\) such that: \[ a - b = kn\]

\textbf{Alternate Definition (\(a\) and \(b\) \(\pmod n\) have same
remainder)}

Often defined :

\[a \equiv b \pmod{n}\] interpreted as: \(a\) and \(b\) , both leave the
\textbf{same remainder} when divided by \(n\), that is: \[
a \pmod{n} = b \pmod{n}
\]

\subsubsection{Why These Definitions Are
Equivalent}\label{why-these-definitions-are-equivalent}

Suppose:

\[a = q_1 n + r \quad \text{and} \quad b = q_2 n + r\]

Then:

\[a - b = (q_1 - q_2)n\] So:

\[n \mid (a - b) \Rightarrow a \equiv b \pmod{n}\] Thus: having the same
remainder mod \(n\) implies \[a \equiv b \pmod{n}\] And conversely, if
\[a \equiv b \pmod{n}\]then \(a\) and \(b\) have the same remainder.

\subsubsection{Why We Sometimes Just
Compute}\label{why-we-sometimes-just-compute}

\[a \mod n\]

When you compute: \[a \mod n = r\] You're finding the
\textbf{representative} of the equivalence class \([a]\): computing
\[a \mod n\]gives the \textbf{least non-negative} value in this class
\#\#\# Summary Table

\begin{longtable}[]{@{}
  >{\raggedright\arraybackslash}p{(\linewidth - 2\tabcolsep) * \real{0.3043}}
  >{\raggedright\arraybackslash}p{(\linewidth - 2\tabcolsep) * \real{0.6957}}@{}}
\toprule\noalign{}
\begin{minipage}[b]{\linewidth}\raggedright
Description
\end{minipage} & \begin{minipage}[b]{\linewidth}\raggedright
Statement
\end{minipage} \\
\midrule\noalign{}
\endhead
\bottomrule\noalign{}
\endlastfoot
Formal definition & \(a \equiv b \pmod{n} \iff n \mid (a - b)\) \\
School interpretation &
\(a \equiv b \pmod{n} \iff a \mod n = b \mod n\) \\
Class representative & \(a \mod n\) gives the label of \([a]\) \\
\end{longtable}

\textbf{Formal Structure} \[a \mod n\] For any integer \(a\) and any
natural number\(n \geq 1\), there exist \textbf{unique integers} \(q\)
and \(r\) such that:

\[ a = qn + r \quad \text{with} \quad 0 \leq r < n\]

where \(q\) is the quotient (how many full times \(n\) fits into \(a\))
ans \(r\) the \textbf{remainder} --- this is what we call \[a \mod n\]
\#\#\# Example: \[a = 17, \ n = 5\]

\[
17 = 3 \cdot 5 + 2 \quad \Rightarrow \quad 17 \mod 5 = 2
\]

So: \[q = 3\] \[r = 2\] \#\#\# Example with Negative Number:
\[a = -7, \ n = 4\]

\[
-7 = (-2) \cdot 4 + 1 \quad \Rightarrow \quad -7 \mod 4 = 1
\]

We choose \[q = -2\] so that the remainder \[r = 1\] satisfies
\[0 \leq r < 4\] \#\#\# Summary

The value of \(a \mod n\) is always the \textbf{least non-negative
remainder} after dividing \(a\) by \(n\). It determines which
\textbf{equivalence class} \(a\) belongs to modulo \(n\)

First, we begin with a set \(S\). We propose a relation \(\sim\) on
\(S\). For illustrative purposes, we define: \[
a \equiv r \mod 4 \iff a - r \in 4\mathbb{Z}
\]

We then \textbf{prove} that this relation is:

\begin{itemize}
\tightlist
\item
  \textbf{Reflexive}
\item
  \textbf{Symmetric}
\item
  \textbf{Transitive}
\end{itemize}

This confirms that \(\sim\) is an \textbf{equivalence relation}. To
analyze the structure, we compute for each element: \[
a \mod 4
\] This value (the \textbf{datum}, or \textbf{tag}) tells us which other
elements \[a\]is related to, and is used to \textbf{label} the
equivalence class:

\[
[a] = \{ x \in S \mid x \equiv a \mod 4 \}
\] We then group all such related elements into \textbf{equivalence
classes}. Finally, the set of all these disjoint classes forms a
\textbf{partition} \(\mathcal{P}_\sim\) of \(S\).

\section{\texorpdfstring{🧮 Example: Constructing
\(\mathbb{Z}/4\mathbb{Z}\) via Equivalence
Classes}{🧮 Example: Constructing \textbackslash mathbb\{Z\}/4\textbackslash mathbb\{Z\} via Equivalence Classes}}\label{example-constructing-mathbbz4mathbbz-via-equivalence-classes}

\begin{center}\rule{0.5\linewidth}{0.5pt}\end{center}

\subsection{🔧 Equivalence Relation}\label{equivalence-relation}

We define a relation on \(\mathbb{Z}\):

\[
a \sim b \iff 4 \mid (b - a)
\]

That is, \(a\) and \(b\) are equivalent if they differ by a multiple of
4.\\
This is the same as saying:

\[
a \equiv b \pmod{4}
\]

\begin{center}\rule{0.5\linewidth}{0.5pt}\end{center}

\subsection{Step 1: Compute Equivalence
Classes}\label{step-1-compute-equivalence-classes}

Each equivalence class \([a]\) contains all integers congruent to \(a\)
mod 4.

\begin{itemize}
\tightlist
\item
  \[ [0] = \{ ..., -8, -4, 0, 4, 8, 12, ... \} \]
\item
  \[ [1] = \{ ..., -7, -3, 1, 5, 9, 13, ... \} \]
\item
  \[ [2] = \{ ..., -6, -2, 2, 6, 10, 14, ... \} \]
\item
  \[ [3] = \{ ..., -5, -1, 3, 7, 11, 15, ... \} \]
\end{itemize}

\begin{center}\rule{0.5\linewidth}{0.5pt}\end{center}

\subsection{Step 2: Define the Quotient
Set}\label{step-2-define-the-quotient-set}

The quotient group is:

\[
\mathbb{Z}/4\mathbb{Z} = \{ [0], [1], [2], [3] \}
\]

These are the \textbf{distinct equivalence classes} under the relation
\(\sim\).

\begin{center}\rule{0.5\linewidth}{0.5pt}\end{center}

\subsection{Step 3: Define the Group
Operation}\label{step-3-define-the-group-operation}

The group operation is addition of classes:

\begin{itemize}
\tightlist
\item
  \[ [1] + [2] = [3] \]
\item
  \[ [2] + [3] = [5] = [1] \]
\item
  \[ [3] + [3] = [6] = [2] \]
\item
  \[ [1] + [1] = [2] \]
\end{itemize}

All arithmetic is performed modulo 4.

\begin{center}\rule{0.5\linewidth}{0.5pt}\end{center}

\subsection{Conclusion}\label{conclusion}

We constructed \(\mathbb{Z}/4\mathbb{Z}\) using the equivalence
relation:

\[
a \sim b \iff a \equiv b \pmod{4}
\]

Each element in the quotient group is an equivalence class of integers
modulo 4, and the group operation is addition modulo 4.

\end{document}
