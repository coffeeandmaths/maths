\documentclass[10pt]{article}

% --------- Minimal page and spacing setup ---------
\usepackage[margin=2cm]{geometry}
\usepackage{setspace}
\setstretch{1.05} % tighter than default
\usepackage{parskip}
\setlength{\parskip}{0.4em}
\setlength{\parindent}{0pt}

% --------- Math and Fonts ----------
\usepackage{amsmath,amssymb,amsfonts}
\usepackage{mathrsfs}  % for script letters
\usepackage{eucal}     % elegant calligraphic fonts
\usepackage{microtype} % better spacing

% --------- Section Style ----------
\usepackage{titlesec}
\titleformat{\section}{\large\bfseries}{\thesection.}{0.5em}{}
\titleformat{\subsection}{\normalsize\bfseries}{\thesubsection}{0.5em}{}

% --------- Header (optional) ----------
\usepackage{fancyhdr}
\pagestyle{fancy}
\fancyhead[L]{\small Notes}
\fancyhead[R]{\small \today}
\fancyfoot[C]{\thepage}

% --------- Document Starts ----------
\begin{document}

\section*{Introduction: From Sets to Mappings as Structure}

A mapping is not merely a process — it is structure, encoded as a subset of a Cartesian product subject to a rule of uniqueness.

Let \( S \) and \( T \) be sets. A mapping \( \tau : S \to T \) may be identified with a subset \( M \subseteq S \times T \) such that for every \( s \in S \), there exists a unique \( t \in T \) with \( (s, t) \in M \). This subset is the \emph{graph} of the mapping, and the condition of uniqueness defines it as a function.

If \( s \tau = t \), we interpret this as: “\( s \) passes through \( \tau \) to yield image \( t \).” This notation emphasizes structure over evaluation, aligning naturally with algebraic and categorical viewpoints.

\vspace{1em}

\textbf{Injectivity} holds when \( s_1 \neq s_2 \) implies \( s_1 \tau \neq s_2 \tau \). \\
\textbf{Surjectivity} means \( \text{Im}(\tau) = T \). \\
\textbf{Bijectivity} combines both — the mapping is invertible.

\end{document}
