\documentclass[12pt]{article}
\usepackage{amsmath, amssymb}
\usepackage{geometry}
\geometry{margin=1in}
\setlength{\parindent}{0pt}


\title{Canonical Decomposition}
\author{Ramon Armeria}
\date{July 2025}

%--- footnotes 
\newcommand{\canonicalnote}{\footnote{A map is said to be \emph{canonical} when it is naturally determined by the structure of the objects involved, with no arbitrary choices.}}


\begin{document}
	
	\maketitle
Let \( A \) and \( B \) be sets, and let \( f \) be a rule
\[
f: A \to B
\]
mapping each \( a \in A \) to a unique element \( f(a) \in B \).\canonicalnote{}  
This rule is \textit{canonical}: once \( f \) is defined, the output for each \( a \) is completely determined.

\medskip

The function does not construct new elements. It \textit{maps} \( a \in A \) to an element in \( B \). The map \( f \) is the mechanism of this pointing.

\medskip

The totality of all such assignments defines the graph of \( f \):
\[
\Gamma_f = \{ (a, f(a)) \mid a \in A \} \subseteq A \times B
\]

This subset \( \Gamma_f \) captures the function entirely:  
it lives inside the Cartesian product \( A \times B \),  
but includes only the pairs \( (a, f(a)) \) where \( f \) assigns a unique output to each input.

In this sense, \( A \times B \) contains all possible input-output combinations,  
and the function \( f \) **selects** exactly one for each \( a \in A \).


\end{document}