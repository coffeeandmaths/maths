\documentclass[12pt]{article}
\usepackage{amsmath, amssymb, amsfonts}
\usepackage{geometry}
\geometry{a4paper, margin=1in}
\usepackage{titlesec}
\usepackage{mathtools}
\usepackage{enumitem}

\titleformat{\section}{\normalfont\Large\bfseries}{\thesection.}{1em}{}
\titleformat{\subsection}{\normalfont\large\bfseries}{\thesubsection.}{1em}{}

\title{\textbf{Abstract Algebraic Structure of \((\mathbb{R}, +)\) and Characters into \(U(1)\)}}
\author{}
\date{}

\begin{document}
	\maketitle
	
	\section{The Additive Group \((\mathbb{R}, +)\)}
	
	We consider the set of real numbers \(\mathbb{R}\) under addition, forming the group \((\mathbb{R}, +)\), which models continuous time shifts in signal theory.
	
	\subsection*{Algebraic Structure}
	
	\begin{itemize}[leftmargin=1.5em]
		\item \textbf{Set}: \(\mathbb{R}\)
		\item \textbf{Binary operation}: \(+\colon \mathbb{R} \times \mathbb{R} \to \mathbb{R}\), defined by \((a, b) \mapsto a + b\)
	\end{itemize}
	
	\subsection*{Group Properties}
	
	\begin{itemize}[leftmargin=1.5em]
		\item \textbf{Abelian Group}: Commutative: \(a + b = b + a\)
		\item \textbf{Identity Element}: \(0 \in \mathbb{R}\), such that \(a + 0 = a\) for all \(a\)
		\item \textbf{Inverse Element}: For every \(a \in \mathbb{R}\), the inverse is \(-a\), with \(a + (-a) = 0\)
		\item \textbf{Infinite}: \(\mathbb{R}\) is uncountably infinite
		\item \textbf{Divisible Group}: For all \(a \in \mathbb{R}\), \(n \in \mathbb{Z}^+\), there exists \(x \in \mathbb{R}\) such that \(nx = a\)
		\item \textbf{Torsion-Free}: \(na = 0\) implies \(a = 0\), for all \(n \in \mathbb{Z}\)
		\item \textbf{Free \(\mathbb{Q}\)-Module}: Scalar multiplication \(q \cdot a\) is well-defined for \(q \in \mathbb{Q}\), and respects module structure
	\end{itemize}
	
	\subsection*{Functional Role in Signals}
	
	\begin{itemize}[leftmargin=1.5em]
		\item Represents \textbf{time translations}: \(x(t) \mapsto x(t + \tau)\)
		\item Defines \textbf{periodicity} via translation symmetry: \(x(t + T) = x(t)\)
	\end{itemize}
	
	\section{The Unit Circle Group \(U(1)\)}
	
	Let us now construct the codomain group for character maps.
	
	\subsection*{Definition}
	
	\[
	U(1) := \{ z \in \mathbb{C} \mid |z| = 1 \}
	\]
	This is the set of all complex numbers of unit modulus. The operation is complex multiplication:
	\[
	\cdot\colon U(1) \times U(1) \to U(1), \quad (z_1, z_2) \mapsto z_1 z_2
	\]
	
	\subsection*{Group Properties}
	
	\begin{itemize}[leftmargin=1.5em]
		\item \textbf{Abelian Group}: \(z_1 z_2 = z_2 z_1\)
		\item \textbf{Identity Element}: \(1 \in \mathbb{C}\), satisfies \(z \cdot 1 = z\)
		\item \textbf{Inverses}: \(z^{-1} = \overline{z}\), since \(z \cdot \overline{z} = |z|^2 = 1\)
		\item \textbf{Infinite}: Uncountably infinite, parameterized by angle \(\theta\) via \(e^{j\theta}\)
	\end{itemize}
	
	\subsection*{Relevance to Signal Theory}
	
	\begin{itemize}[leftmargin=1.5em]
		\item Represents \textbf{phase}, \textbf{rotation}, and \textbf{frequency content}
		\item Hosts the values of \textbf{complex exponentials}: \(e^{j \omega t} \in U(1)\)
	\end{itemize}
	
	\section{Characters: Homomorphisms from \((\mathbb{R}, +)\) to \(U(1)\)}
	
	We define a family of functions (called \textbf{characters}):
	
	\[
	\chi_\omega\colon \mathbb{R} \to U(1), \quad t \mapsto e^{j \omega t}
	\]
	
	\subsection*{Homomorphism Verification}
	
	For all \(t_1, t_2 \in \mathbb{R}\):
	
	\[
	\chi_\omega(t_1 + t_2) = e^{j \omega (t_1 + t_2)} = e^{j \omega t_1} \cdot e^{j \omega t_2} = \chi_\omega(t_1) \cdot \chi_\omega(t_2)
	\]
	
	Thus, each \(\chi_\omega\) is a group homomorphism from \((\mathbb{R}, +)\) into \((U(1), \cdot)\).
	
	\subsection*{Interpretation}
	
	Each character \(\chi_\omega\) encodes:
	
	\begin{itemize}[leftmargin=1.5em]
		\item A rotation on the unit circle at angular frequency \(\omega\)
		\item A continuous symmetry of time under translation
		\item A building block for sinusoidal signals via Euler's identity:
		\[
		e^{j \omega t} = \cos(\omega t) + j \sin(\omega t)
		\]
	\end{itemize}
	
	\section*{Conclusion}
	
	We have formally described the group \((\mathbb{R}, +)\) as the structure underlying continuous time, and constructed the unit circle group \(U(1)\) as the natural codomain for periodic signal representations. The map \( t \mapsto e^{j \omega t} \) arises canonically as a character (group homomorphism), embedding sinusoidal behavior directly into algebraic structure.
	
	Let \( x \colon \mathbb{R} \to U(1) \) be defined by:
	\[
	x(t) := e^{j t}
	\]
	
	We claim that \( x \) is a group homomorphism:
	\[
	x(t_1 + t_2) = e^{j(t_1 + t_2)} = e^{j t_1} \cdot e^{j t_2} = x(t_1) \cdot x(t_2)
	\]
	
	Therefore, \( x \in \mathrm{Hom}((\mathbb{R}, +), (U(1), \cdot)) \), i.e., a character of \( \mathbb{R} \).
	
	\section{The Unit Circle Group \( U(1) \)}
	
	\subsection*{Definition}
	
	The \textbf{unit circle group} \( U(1) \) is defined as the following subset of the complex numbers:
	\[
	U(1) := \{ z \in \mathbb{C} \mid |z| = 1 \}
	\]
	Each element \( z \in U(1) \) is a complex number of modulus 1, i.e., it lies on the unit circle in the complex plane. Every such element can be written in polar (Euler) form as:
	\[
	z = e^{j \theta} = \cos(\theta) + j \sin(\theta), \quad \theta \in \mathbb{R}
	\]
	
	\subsection*{Group Structure}
	
	The group operation is complex multiplication:
	\[
	\cdot\colon U(1) \times U(1) \to U(1), \quad (z_1, z_2) \mapsto z_1 z_2
	\]
	
	\begin{itemize}
		\item \textbf{Closure}: If \( |z_1| = 1 \) and \( |z_2| = 1 \), then \( |z_1 z_2| = |z_1||z_2| = 1 \Rightarrow z_1 z_2 \in U(1) \)
		\item \textbf{Associativity}: Inherited from complex multiplication
		\item \textbf{Identity element}: \( 1 \in \mathbb{C} \) satisfies \( z \cdot 1 = z \)
		\item \textbf{Inverse}: For any \( z \in U(1) \), the inverse is \( z^{-1} = \overline{z} \), since \( z \overline{z} = |z|^2 = 1 \)
		\item \textbf{Commutativity}: \( z_1 z_2 = z_2 z_1 \)
	\end{itemize}
	
	Thus \( U(1) \) is an \textbf{abelian group} under multiplication.
	
	\subsection*{Geometric Interpretation}
	
	Each \( z \in U(1) \) represents a unit vector from the origin in the complex plane, forming an angle \( \theta \) with the real axis. The operation of multiplying by \( z \) corresponds to a rotation.
	
	\paragraph{Let \( a \in \mathbb{C} \), and \( z = e^{j\theta} \in U(1) \):}
	\[
	z \cdot a = e^{j\theta} a
	\]
	
	This operation:
	\begin{itemize}
		\item Preserves the magnitude of \( a \): \( |z a| = |z||a| = |a| \)
		\item Rotates \( a \) by angle \( \theta \): \( \arg(z a) = \arg(a) + \theta \)
	\end{itemize}
	
	Therefore, multiplication by an element of \( U(1) \) is equivalent to a \textbf{rotation transformation} on the complex plane.
	
	\subsection*{Application in Homomorphisms}
	
	The unit circle group \( U(1) \) is the natural codomain for homomorphisms from \( (\mathbb{R}, +) \), such as:
	\[
	x(t) = e^{j t} \in U(1)
	\]
	
	Here, \( t \mapsto e^{j t} \) maps real-valued time to **pure phase rotations** on the unit circle, encoding periodic behavior.
	
	
	
	
\end{document}
